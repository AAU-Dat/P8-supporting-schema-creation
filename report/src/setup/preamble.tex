%! Author = Runge
%! Date = 29-12-2023

% Packages
\RequirePackage{clrscode4e}
\usepackage{microtype}
\usepackage{mathtools}
\usepackage[notext]{stix2}
\usepackage{booktabs}
\usepackage{tikz}
\usepackage{amsmath}
\usepackage{thm-restate}

% Packages with options set
\usepackage[inference]{semantic}
\usepackage[hidelinks]{hyperref}
\usepackage[textsize=small,obeyDraft]{todonotes}
\usepackage[newfloat, outputdir=../out]{minted}
%\usepackage[autostyle]{csquotes}
\usepackage[backend=biber,
    bibencoding=utf8,
    maxbibnames=20,
    style=ieee,
    citestyle=numeric-comp,
    url=false
]{biblatex}

\usepackage{pf2}
\usepackage{orcidlink}

% Package setup
\makeatletter
\hypersetup{%
    plainpages=false,%
    pdftitle=\@title,%
    pdfauthor={Anders Malta Jacobsen, Casper Ståhl, Daniel Runge Petersen, Lars Emanuel Hansen, Oliver Holmgaard, Sebastian Aaholm},%
    pdflang={en-GB},%
    pdfsubject={Semester project at Aalborg University},%
    pdfkeywords ={Abstract Interpretation, Databases, Formal Verification, Program Analysis}
    bookmarksnumbered=true,%
    colorlinks=true,%
    citecolor=black,%
    filecolor=black,%
    linkcolor=black,% you should probably change this to black before printing
    urlcolor=black,%
    pdfstartview=FitH%
}
\makeatother

\setlength{\marginparwidth}{2cm} % todonotes width
\setminted{linenos, autogobble, breaklines, fontsize=\footnotesize, style=friendly, xleftmargin=1em, numbersep=5pt}
\usemintedstyle[bash]{emacs}
\addbibresource{bib/main.bib}

% Other setup and options
\declaretheorem{theorem}
\declaretheorem{lemma}
\declaretheorem{corollary}
\declaretheorem{definition}
\declaretheorem{example}
\declaretheorem{conjecture}
\DeclareFloatingEnvironment[name=Algorithm, placement=htb!]{algorithm}
%\newfloat{algorithm}{htb!}{lop}
%\floatname{algorithm}{Algorithm}
%\newcommand{\algorithmautorefname}{Algorithm}

\makeatletter
\providecommand{\bigsqcap}{%
  \mathop{%
    \mathpalette\@updown\bigsqcup
  }%
}
\newcommand*{\@updown}[2]{%
  \rotatebox[origin=c]{180}{$\m@th#1#2$}%
}
\makeatother


\usetikzlibrary{shapes, arrows, fit, calc, automata, positioning}

\newcommand{\abssem}[1]{\ab{\mathcal{S}} \lBrack #1 \rBrack}
\newcommand{\sem}[1]{\mathcal{S} \lBrack #1 \rBrack}
\newcommand{\env}[1]{\rho_{#1}}
\newcommand{\abstablep}{t'^{\#}}
\newcommand{\absvars}{\ab{a}}
\newcommand{\absattrs}{\mathbf{v}_d}
\newcommand{\absattr}{v_d}
\newcommand{\absexps}{\mathbf{e}}
\newcommand{\absexp}[1]{e_{#1}}
\newcommand{\abspred}{\phi}
\newcommand{\clattice}[2]{C_{#1}(#2)}
\newcommand{\concrete}{\gamma}
\newcommand{\aabstract}{\alpha}
\DeclareMathOperator{\map}{map}
\newcommand{\absboolsem}[1]{\widehat{\mathcal{B}}\lBrack #1 \rBrack}
\newcommand{\boolsem}[1]{\mathcal{B}\lBrack #1 \rBrack}
\newcommand{\absexpsem}[1]{\widehat{\mathcal{E}}\lBrack #1 \rBrack}
\newcommand{\expsem}[1]{\mathcal{E}\lBrack #1 \rBrack}
\newcommand{\true}{t\!t}
\newcommand{\false}{f\!\!f}
\newcommand{\unknown}{?\!?}
\newcommand{\ab}[1]{\widehat{#1}}
\newcommand{\aab}[1]{\widetilde{#1}}
\newcommand{\aaab}[1]{\widetilde{\widetilde{#1}}}
\newcommand{\aop}{\;\ab{op}\;}
\newcommand{\op}{\;op\;}
\newcommand{\ap}[1]{\;\langle#1\rangle\;}
\newcommand{\aaop}{\;\aab{op}\;}
\newcommand{\mnull}{null}
\newcommand{\abstable}{\ab{t}}
\newcommand{\comp}{comp}
\newcommand{\into}{\hookrightarrow}
\newcommand{\lookupcl}{\mathcal{C}}
\newcommand{\abssqlsem}[1]{\widehat{\mathcal{Q}}\lBrack #1 \rBrack}
\newcommand{\sqlsem}[1]{\mathcal{Q}\lBrack #1 \rBrack}
\newcommand{\abspredsem}[1]{\widehat{\mathcal{W}}\lBrack #1 \rBrack}
\newcommand{\predsem}[1]{\mathcal{W}\lBrack #1 \rBrack}
\newcommand{\absatomsem}[1]{\widehat{\mathcal{T}}\lBrack #1 \rBrack}
\newcommand{\app}[1]{\;\left \lParen#1\right \rParen\;}
\newcommand{\regexs}{\mathbf{REG}}
\newcommand{\regex}{R}
\newcommand{\lang}{\mathcal{L}}
\newcommand{\uints}{\mathbf{INT}}
\newcommand{\uint}{\mathscr{I}}
\newcommand{\strs}{\mathbb{S}}
\newcommand{\nums}{\mathbb{Z}}
\newcommand{\bools}{\mathbf{BOOL}}
\newcommand{\instructions}{\mathbb{I}}
\newcommand{\environments}{\mathfrak{E}}
\newcommand{\environment}{\rho}
\newcommand{\expressions}{\mathbb{E}}
\newcommand{\powerset}[1]{\mathcal{P}\left(#1\right)}
\newcommand{\powerbag}[1]{\mathcal{M}\left(#1\right)}
\newcommand{\cc}{c}
\newcommand{\strings}{\mathbf{STR}}
\newcommand{\integers}{\mathbf{NUM}}
\DeclareMathOperator{\unaryoperator}{op_u}
\DeclareMathOperator{\binaryoperator}{op_b}
\DeclareMathOperator{\compareoperator}{comp}
\floatplacement{listing}{htb!}
\floatplacement{table}{htb!}
\floatplacement{figure}{htb!}
\floatplacement{algorithm}{htb!}
