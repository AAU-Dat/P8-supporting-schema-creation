%! Author = Runge
%! Date = 29-12-2023

% Packages
\RequirePackage{clrscode4e}
\usepackage{amsmath}
% \usepackage{amsthm} \\ the proof environment clashes with the one in pf2.
\usepackage{amssymb}
\usepackage{amsbsy}
\usepackage{mathrsfs}
\usepackage{dsfont}
\usepackage{bbold}
\usepackage{booktabs}
\usepackage{tikz}
\usepackage{pf2}
\usepackage{thmtools, thm-restate}
\usepackage{stmaryrd}

% Packages with options set
\usepackage[hidelinks]{hyperref}
\usepackage[textsize=small,obeyDraft]{todonotes}
\usepackage[newfloat]{minted}
\usepackage[backend=biber,
    bibencoding=utf8,
    maxbibnames=20,
    style=ieee,
    citestyle=numeric-comp,
    url=false
]{biblatex}
\usepackage[acronym]{glossaries}

% Package setup
\setlength{\marginparwidth}{2cm} % todonotes width
\setminted{linenos, autogobble, breaklines, fontsize=\footnotesize, style=friendly, xleftmargin=1em, numbersep=5pt}
\addbibresource{bib/main.bib}

% Other setup and options
\declaretheorem{theorem}
\declaretheorem{lemma}
\declaretheorem{definition}
\declaretheorem{example}
\declaretheorem{conjecture}
\newfloat{algorithm}{htb!}{lop}
\floatname{algorithm}{Algorithm}
\newcommand{\algorithmautorefname}{Algorithm}

\makeatletter
\providecommand{\bigsqcap}{%
  \mathop{%
    \mathpalette\@updown\bigsqcup
  }%
}
\newcommand*{\@updown}[2]{%
  \rotatebox[origin=c]{180}{$\m@th#1#2$}%
}
\makeatother

\makeglossaries

\tikzstyle{state} = [rectangle, minimum width=1.5cm, minimum height=1cm, text centered, draw=black, fill=white!30]
\tikzstyle{circle} = [ellipse, minimum width=1.5cm, minimum height=0.5cm, text centered, draw=black, fill=white!30]

\newcommand{\abssem}[1]{S^\# \llbracket #1 \rrbracket}
\newcommand{\absexpsem}[1]{E^\# \llbracket #1 \rrbracket}
\newcommand{\env}[1]{\rho_{#1}}
\newcommand{\abstable}{t^\#}
\newcommand{\abstablep}{t'^{\#}}
\newcommand{\absvars}{a^\#}
\newcommand{\absattrs}{\overset{\rightarrow}{v_d^\#}}
\newcommand{\absattr}{x^\#}
\newcommand{\absexps}{\overset{\rightarrow}{e^\#}}
\newcommand{\absexp}[1]{e^\#_{#1}}
\newcommand{\abspred}{\phi^\#}
\newcommand{\clattice}[2]{C_{#1}(#2)}
\newcommand{\concrete}{\gamma}
\newcommand{\aabstract}{\alpha}
\DeclareMathOperator{\map}{map}
