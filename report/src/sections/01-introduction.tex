\section{Introduction}\label{sec:introduction}
\IEEEPARstart{S}{oftware} analysis is a broad field encompassing various techniques and methodologies, each tailored to address specific aspects of software systems.
Static analysis is one such technique, focusing on finding information guaranteed to hold at all executions of a program~\cite{jackson_software_2000}.
Data flow analysis is a common approach within static analysis, used to examine the data flow through a program.
In this paper, we employ static program analysis, specifically abstract interpretation, a specialized form of data flow analysis, to analyze software behaviour.

%Two common approaches to sound analysis are dynamic and static analysis.
%Dynamic analysis finds information guaranteed to hold at a specific execution, while static analysis involves finding information guaranteed to hold at all executions.
%While the dynamic approach offers valuable insights into the specific behaviour of a program's single execution, static analysis provides a more comprehensive view by ensuring that certain properties hold in all possible executions~\cite{jackson_software_2000}.

Abstract interpretation, first proposed by Cousot and Cousot in 1977~\cite{cousot_abstract_1977}, has emerged as a fundamental method for static program analysis.
Over the years, it has evolved into a versatile tool, finding applications across various programming paradigms and system architectures.
Its utility extends beyond traditional imperative programming to encompass object-oriented designs and concurrent systems~\cite{gustafsson_analyzing_2013, mine_static_2023}.
Abstract interpretation has played a pivotal role in analyzing a spectrum of program properties, from basic safety and liveness concerns to intricate security considerations~\cite{mastroeni_abstract_2011}.

Building upon this, Halder and Cortesi have expanded the application of abstract interpretation to domains like query languages~\cite{halder_abstract_2012}.
This paper continues in these footsteps, aiming to further explore and extend the capabilities of abstract interpretation within the context of query languages.


\subsection{Non-terminating Analysis}\label{subsec:non-terminating-analysis}
The framework proposed by Halder and Cortesi may encounter non-termination issues.
While it can used to optimize integrity constraints through static analysis of transactions, it does not specify particular properties to check.
To illustrate, consider the database schema in \autoref{lst:motivate-sql}, the insert query in \autoref{lst:motivate-sql-insert}, and the program in \autoref{lst:motivate-program}.
The program inserts a tuple into the account table with the name "name" and a balance of 1 into the account table, and it does so indefinitely.


\begin{listing}
    \begin{minted}{sql}
        CREATE TABLE account (
            name    TEXT,
            balance INT
        );
    \end{minted}
    \caption{A simple schema representing an account.}
    \label{lst:motivate-sql}
\end{listing}


\begin{listing}
    \begin{minted}{bash}
        while true do
            insert((name, balance), ("name", 1))
    \end{minted}
    \caption{A tiny program with nonterminating analysis.}
    \label{lst:motivate-program}
\end{listing}

\begin{listing}
    \begin{minted}{sql}
    INSERT INTO account VALUES ("name", 1);
    \end{minted}
    \caption{A simple SQL statement inserting a tuple into the account table.}
    \label{lst:motivate-sql-insert}
\end{listing}


Constantly changing database contexts under Cortesi and Halder's framework may lead to analysis termination issues.
However, there is a solution.
By extending the analysis to include abstract tuples, i.e. representing tuples with abstract values and abstracting over the multiplicity in the table, the analysis can terminate even for the program in \autoref{lst:motivate-program}.


\todo{\textbf{TODO:Show the contributions through a second example that runs through how we used the updated semantics. Move from method}
}

\subsection{Related Work}\label{subsec:related-work}
This paper builds on the seminal work of abstract interpretation by Cousot and Cousot~\cite{cousot_abstract_1977}, and in particular we consider the follow-up paper~\cite{cousot_abstract_1996} from which we take the idea of using intervals and regular expressions as the abstract domains in our analysis.

Additionally, we heavily use the work of abstract interpretation of database queries by Cortesi and Halder~\cite{halder_abstract_2012}.
Concretely, we use their work for analysing a database schema in relation to a program using the syntax they describe.

Jana et al.~\cite{jana_extending_2020} have extended the application of abstract interpretation to the domain of web security.
Dependency information (data- and/or control-dependencies) among program variables and program statements is playing crucial roles in a wide range of software-engineering activities, e.g., program slicing, information flow security analysis, debugging, code optimization, code reuse, code understanding.
Most existing dependency analyzers focus on mainstream languages, and they do not support database applications embedding queries and data-manipulation commands.
This paper extends the Abstract Interpretation framework for static dependency analysis of database applications, providing a semantics-based computation tunable with respect to precision.

Kashyap et al.~\cite{kashyap_integrity_2022} have proposed a new approach to integrity checking of database applications using abstract interpretation.
The approach is based on the use of abstract interpretation to analyze the database application and to check the integrity of the data, i.e., to verify that the data is consistent with the constraints defined in the database management system.


\subsection{Contributions}\label{subsec:contributions}
As described in \autoref{subsec:related-work}, one of our main contributions is describing an analysis made by extending the work made by Cortesi and Halder.

Included in this, we extend the level of abstraction used in the analysis.
Where~\cite{halder_abstract_2012} only abstract over values, we abstract over both the tuples and values, and the multiplicity of the tuples and values in the database, ultimately guaranteeing a finite analysis.
This is thoroughly covered in \autoref{sec:theory}.

Also included in the contribution is a description of an analysis used to check properties when given a program and its associated database.
While we have not provided a specific method for checking general properties, we have outlined a systematic approach for verifying invariants.
Invariants are a particular type of property that should always be held during the program's execution.

This way, we can run the analysis on the program and afterward check at each step of the program if the invariant holds.
If it does not, we can stop the analysis and report that the program is incorrect.
We can check for invariants in every state of the program.
We do this in a program graph, where each node represents a state of the program.
We can check if the invariant holds in each node.
This way, we can systematically verify the program's correctness throughout its execution.

The invariants can be checked with the predicate syntax which is described in \autoref{fig:abstract-syntax}.

To ensure that our work is correct, we informally prove soundness of the analysis.
By doing this, we can always be sure that our analysis produces an overestimation of the concrete values and tuples.
The soundness proof can be found in \autoref{subsec:soundness}.

\subsection{Article Overview}\label{subsec:article-overview}
The rest of the article is structured as follows:
In \autoref{sec:preliminaries} we cover the preliminaries needed to understand the rest of the paper, in particular we cover lattices, program graphs and their semantics and abstract interpretation.
In \autoref{sec:theory} we describe our contribution in detail:

\begin{itemize}
    \item In \autoref{subsec:abstract-syntax} we describe the syntax of the language we of the analysis, further we define the conversion of programs in this language to program graphs;
    \item in \autoref{subsec:abstract-domains} we describe the abstract domains of our analysis and how they related to the concrete domains, notably we describe:
    \begin{itemize}
        \item Regular expressions as an abstract domain for strings,
        \item union intervals as an abstract domain for numbers,
        \item cover-lattices as a technique to deal with infinite lattices in a finite manner,
        \item Abstract bags of abstract tuples as an abstract domain of bags.
    \end{itemize}
    \item In \autoref{subsec:abstract-semantics} we give the abstract semantics for the instructions pertaining to the edges of program graphs, of note we define the abstract semantics for the SQL select-, insert-, update- and delete statement;
    \item In \autoref{subsec:soundness} we argue that our analysis is sound;
    \item and finally in \autoref{} we show that our analysis will terminate.\todo{Write this section/subsection}
\end{itemize}

In \autoref{sec:discussion} we discuss the quality of the presented analysis and in \autoref{sec:discussion} we also present new avenues for future work.
Finally the we conclude the paper in \autoref{sec:conclusion}.



