%! Author = Runge
%! Date = 29-12-2023

\section{Introduction}\label{sec:introduction}
%background and motivation
Abstract interpretation, first proposed by Cousot and Cousot in 1977~\cite{cousot_abstract_1977}, has emerged as a fundamental method for static program analysis.
Over the years, it has evolved into a versatile tool, finding applications across various programming paradigms and system architectures.
Its utility extends beyond traditional imperative programming to encompass object-oriented designs and concurrent systems~\cite{gustafsson_analyzing_2013, mine_static_2023}.
Abstract interpretation has played a pivotal role in analyzing a spectrum of program properties, from basic safety and liveness concerns to intricate security considerations~\cite{mastroeni_abstract_2011}.


Building upon this historical foundation, subsequent researchers, such as Halder and Cortesi, have expanded the application of abstract interpretation to domains like query languages \cite{halder_abstract_2012}.
This paper continues in the footsteps of these pioneers, aiming to further explore and extend the capabilities of abstract interpretation within the context of query languages.

%Objective
\todo[inline]{I think we should be more specific about the objective. This is not done}
The objective of this work is to develop a general software tool, in the sense of value analysis, only be considering abstract values, not meta information like information flow or taint analysis.
Within this limitation we will restrict try to pick the most abstract/powerfull primitives possible, that do not encounter computation problems.
Utilizing abstract interpretation for analyzing the reachable state space of an entity relation database given its schema, accompanying reactive procedures and a model of the behavior of the environment in which the database will reside.


\todo[inline]{Maybe we can use workflow nets for the environment behavior model.}

This research endeavor is motivated by the observation that, despite the widespread use of abstract interpretation in program analysis, there is a notable absence of a comprehensive, general-purpose tool tailored specifically for value analysis in query languages.
The lack of such a tool represents a significant gap in the field, hindering researchers and practitioners from effectively analyzing and verifying complex systems.
By addressing this gap, our work seeks to provide a valuable contribution to the field of program analysis, enabling more efficient and accurate evaluation of software systems.
\paragraph{Justification}

It seems that no general tool has been developed.


\section{Notes}
\paragraph{Guidance to the reader}

\begin{itemize}
    \item The reader should have a basic understanding of SQL.
    \item The reader should understand abstract interpretation and whatever semantics we are using.
    \item The reader should understand that a main contribution is a open source tool.
\end{itemize}

\paragraph{Conclusion}

\begin{itemize}
    \item A software tool of limited scope has been developed.
    \item The abstractions and concretizations from a Galios connection
    \item The analysis is show to be sound.
    \item The usability of the system is discussed but not tested with rigor. 
\end{itemize}







