\section{Introduction}\label{sec:introduction}
Abstract interpretation, first proposed by Cousot and Cousot in 1977~\cite{cousot_abstract_1977}, has emerged as a fundamental method for static program analysis.
Over the years, it has evolved into a versatile tool, finding applications across various programming paradigms and system architectures.
Its utility extends beyond traditional imperative programming to encompass object-oriented designs and concurrent systems~\cite{gustafsson_analyzing_2013, mine_static_2023}.
Abstract interpretation has played a pivotal role in analyzing a spectrum of program properties, from basic safety and liveness concerns to intricate security considerations~\cite{mastroeni_abstract_2011}.

Building upon this, Halder and Cortesi have expanded the application of abstract interpretation to domains like query languages~\cite{halder_abstract_2012}.
This paper continues in these footsteps, aiming to further explore and extend the capabilities of abstract interpretation within the context of query languages.

Software analysis is a broad field encompassing a wide range of techniques and methodologies, each tailored to address specific aspects of software systems.
One such technique is value analysis, which focuses on the behavior of a program with respect to its input and output values~\cite{jackson_software_2000}.
Value analysis can be approached from various angles, including the consideration of meta-information like information flow or taint analysis.
While these approaches offer valuable insights into the behavior of a program, they often require detailed knowledge of the program's internal workings and are not always applicable to all types of software systems.

The 