%! Author = Runge
%! Date = 29-12-2023

\section{Proof of theorem \ref{thm:partition}}

\partition*

<<<<<<< HEAD
\reglattice*
\prove{
    $(REG, \subseteq)$ is a partial order, and $\cup$ and $\cap$ are the join and meet respectively.
    $\bigcup$ or $\bigcap$ is not the lup and glb respectively.
}
\begin{proof}
    \pf\
    \step{poset}{$(REG, \subseteq)$ is a partial order.}
    \begin{proof}
        \pf\ Know property.
    \end{proof}
    \step{cup}{
        $\forall R_1, R_2 \in REG : R_1 \cup R_2 \in REG \land \{R_1, R_2\} \subseteq R_1 \cup R_2 \land \forall R \in REG : \{R_1, R_2\} \subseteq R \implies R_1 \cup R_2 \subseteq R$.
    }
    \begin{proof}
        \pf\ Obvious, $REG$ is closed under finite union.
    \end{proof}
    \step{cap}{$\forall R_1, R_2 \in REG : R_1 \cap R_2 \subseteq \{R_1, R_2\} \land \forall R \in REG : R \subseteq \{R_1, R_2\} \implies R \subseteq R_1 \cap R_2$.}
    \begin{proof}
        \pf\ Obvious, $REG$ is closed under finite intersection.
    \end{proof}
    \step{notbigcup}{$\neg\forall \mathcal{R} \subseteq REG : \bigcup \mathcal{R} \in REG \land \dots$}
    \begin{proof}
        \pf\ $REG$ is not closed under transfinite union.
    \end{proof}
    \qedstep{}
    \begin{proof}
        \pf\ By \stepref{poset}, \stepref{cup}, \stepref{cap} and \stepref{notbigcup}.
    \end{proof}
\end{proof}

Proofs of Theorem \ref{thm:finite-reg-lattice} and \ref{thm:reg-partition-lattice} are similar to the one above, but $\bigcup \mathcal{R}$ can be shown to be defined for all $\mathcal{R} \subseteq \mathcal{REG}$ where $\mathcal{REG} \in \{p(REG^\subset), REG^{\leq n}\}$, as these particular sets are finite.
=======
\begin{proof}
    \pf\
    \assume{
        $(S, \sqsubseteq, \sqcup, \sqcap)$ is a non-complete lattice and $X \subset S$ and $\bigsqcup X = \top$.
    }
    \prove{
        $(p(X), \sqsubseteq, \sqcup, \sqcap) \text{is a complete lattice}$.
    }
    \step{complete}{A nonempty finite lattice is a complete lattice.}
    \begin{proof}
        \pf\ Know property \cite{moller_statitc_nodate}.
    \end{proof}
    \step{lattice}{$(p(X), \sqsubseteq, \sqcup, \sqcap)$ is a lattice.}
    \begin{proof}
        \step{poset}{$p(X)$ is a poset}
        \begin{proof}
            \step{poset-subset}{The subset of a poset is a poset under the same ordering relation.}
            \begin{proof}
                \pf\ Know property.\todo{source missing}
            \end{proof}
            \step{partition-subset}{$p(X) \subset S$}
            \begin{proof}
                \pf\ By the definition of a partition and lattices.
            \end{proof}
            \qedstep
            \begin{proof}
                \pf\ By \stepref{poset-subset} and \stepref{partition-subset}.
            \end{proof}
        \end{proof}
        \step{capcup}{$\forall x, y \in p(X) : x \sqcup y \in p(X) \land x \sqcap y \in p(X)$}
        \begin{proof}
            \pf\ By the definition of $p(X)$.
        \end{proof}
        \qedstep
        \begin{proof}
            \pf\ By \stepref{poset} and \stepref{capcup}.
        \end{proof}
    \end{proof}
    \step{finite}{$p(X)$ is finite.}
    \begin{proof}
        \pf\ there is a finite number of ways to combine the elements of $X$ using $\sqcap$ and $\sqcup$ without repeating operations and repeated operations are idempotent, therefore $p(X)$ is finite.
    \end{proof}
    \qedstep
    \begin{proof}
        \pf\ By \stepref{complete}, \stepref{lattice} and \stepref{finite}.
    \end{proof}
\end{proof}

>>>>>>> 64e4125ab00c4fb504ddea70aeff50c8d947c654

