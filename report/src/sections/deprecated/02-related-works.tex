\section{Related Works}\label{sec:related-works}
Several other papers have been published on the topic of abstract interpretation, but they are not directly related to our work.
Papers such as Halder and Cortesi~\cite{halder_abstract_2012} have expanded the application of abstract interpretation to domains like recursive queries in databases.
recursive queries are a powerful feature of modern database management systems, and they are used to express complex computations in a compact and elegant way.
The authors have proposed a new approach to the static analysis of recursive queries using abstract interpretation.
The approach is based on the use of abstract interpretation to analyze the recursive query and to infer properties of the query results.
While our work does not directly address recursive queries, the exploration of this topic in Halder and Cortesi's paper is intriguing and suggests avenues for future research.


Jana et al.~\cite{jana_extending_2020} have extended the application of abstract interpretation to the domain of web security.
Dependency information (data- and/or control-dependencies) among program variables and program statements is playing crucial roles in a wide range of software-engineering activities, e.g., program slicing, information flow security analysis, debugging, code optimization, code reuse, code understanding.
Most existing dependency analyzers focus on mainstream languages, and they do not support database applications embedding queries and data-manipulation commands.
This paper extends the Abstract Interpretation framework for static dependency analysis of database applications, providing a semantics-based computation tunable with respect to precision.

Kashyap et al.~\cite{kashyap_integrity_2022} have proposed a new approach to integrity checking of database applications using abstract interpretation.
The approach is based on the use of abstract interpretation to analyze the database application and to check the integrity of the data, i.e., to verify that the data is consistent with the constraints defined in the database management system.
