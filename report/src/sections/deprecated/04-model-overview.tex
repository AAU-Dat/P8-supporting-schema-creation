\subsection{Model Overview}\label{subsec:model-overview}

In the following sections, we will be giving an overview of the model we are implementing.

\subsubsection{Overall model}\label{subsubsec:overall-model}

This is a general overview of our model

\begin{algorithm}[htb!]
    \begin{codebox}
        \Procname{$\proc{SQAAL}(S, P, A)$}
        \li $G = programGraph(A)$
        \li $ass = analyze(G,S)$
        \li $result = check(ass,P)$
        \li \Return $result$
    \end{codebox}
    \caption{General model}
    \label{alg:model}
\end{algorithm}
\noindent where, $S$ is the database schema, $P$ are the properties, and $A$ is the application in which the database is running.

\subsubsection{Program graph}\label{subsubsec:program-graph}

We will now describe how to make a graph of the state space

\begin{algorithm}[htb!]
    \begin{codebox}
        \Procname{$\proc{programGraph}(A)$}
        \li TODO
    \end{codebox}
    \caption{Construction of program graph}
    \label{alg:program-graph}
\end{algorithm}

\subsubsection{Analysis}\label{subsubsec:analysis}

We will now describe how our analysis works.
Here we create our state space, represented as a transition system.
This allows us to do model checking

\begin{algorithm}[htb!]
    \begin{codebox}
        \Procname{$\proc{Analysis}(G,S)$}
        \li $todo \coloneqq G.initial\_state$
        \li $state\_space \coloneqq \emptyset$
        \li \While{$todo \neq \emptyset$}
        \li \Do
        $(s,\rho_d,\rho_a) \coloneqq pop(todo)$
        \li \For{$\xrightarrow[]{\text{i}}s'\in s$}
        \li \Do $(\rho'_d,\rho'_a) \coloneqq \widehat{\mathcal{S}}\lBrack i\rBrack(\rho_d,\rho_a)$
        \li \If{$(s',\rho'_d,\rho'_a)$ \textbf{not in} $state\_space$}
        \li \Do $Insert(s',\rho'_d,\rho'_a,todo)$
        \End
        \End
        \li $Insert(s,\rho_d,\rho_a,state\_space)$
        \End
        \End
        \End
        \End
        \li \Return $state\_space$
    \end{codebox}
    \caption{Analysis function}
    \label{alg:analysis}

\end{algorithm}

\subsubsection{Check}\label{subsubsec:check}

We will now describe how we check different properties.
For this, we will be using Linear Temporal Logic to determine whether a property holds.

\begin{algorithm}[htb!]
    \begin{codebox}
        \Procname{$\proc{Check}(ass, P)$}
        \li Model check on state space/transition system
        \zi with LTL properties
        \li // Dismiss the following for now
        \li $failed\_prop \coloneqq []$
        \li \For{$p$ \textbf{in} $P$}
        \li \Do \For{$s$ \textbf{in} ass$$}
        \li \Do \If{$p$ \textbf{not holds in} $s$}
        \li \Do $failed\_prop.insert(p)$
        \End
        \End
        \li \textbf{break}
        \End
        \End
        \End
        \End
        \li \Return $failed\_prop$
    \end{codebox}
    \caption{Check with properties}
    \label{alg:check}
\end{algorithm}
