\section{Proof of theorem \ref{thm:partition}}

\begin{lemma}\label{lem:finite-partition}
    For a non-complete lattice $(S, \sqsubseteq, \sqcup, \sqcap)$ and a finite non-empty subset $X \subseteq S$ where $\bigsqcup X = \top$, $p_X(S)$ is finite.
\end{lemma}

\begin{proof}
    \pf\
    \step{base-case}{\case{$X = \{x\}$}}
    \begin{proof}
        \step{}{$x = \top$}
        \begin{proof}
            \pf\ By the assumption $\bigsqcup \{x\} = \top$.
        \end{proof}
        \step{}{$p_X(S) = \{\top\}$}
        \begin{proof}
            \pf\ By the definition of $p$: $\top \in p_X(S)$, $\top \sqcup \top = \top \in p_X(S)$ and $\top \sqcap \top = \top \in p_X(S)$.
        \end{proof}
    \end{proof}
    \step{inductive-case}{\case{$|X| > 1$}}
    \begin{proof}
        \assume{$p_X(S)$ is finite.}
        \prove{$p_{(X \cup \{y\})}(S)$ is finite, when $y \in S \setminus X$.}
        \pf\ It should be obvious that $\forall x \in p_X(S) : x \in p_{(X \cup \{y\})}(S)$.
        By the definition of $p$ $\forall x \in p_X(S) : x \sqcup y \in p_{(X \cup \{y\})}(S)$ (analogous for $\sqcap$).
        Further $(x \sqcup y) \sqcup y = x \sqcup y$ (analogous for $\sqcap$), and $(x \sqcup y) \sqcap y = y$, and $(x \sqcap y) \sqcup y = y$.
        This means that $\forall x \in p_X(S) : x \in p_{(X \cup \{y\})}(S) \land (x \sqcup y) \in p_{(X \cup \{y\})}(S)$ and no more elements are in $p_{(X \cup \{y\})}(S)$, thus $p_{(X \cup \{y\})}(S)$ is finite.
    \end{proof}
    \qedstep
    \begin{proof}
        \pf\ By induction, \stepref{base-case} and \stepref{inductive-case}.
    \end{proof}
\end{proof}

\partition*

\begin{proof}
    \pf\
    \assume{
        $(S, \sqsubseteq, \sqcup, \sqcap)$ is a non-complete lattice and $X \subset S$ and $\bigsqcup X = \top$.
    }
    \prove{
        $(p_X(S), \sqsubseteq, \sqcup, \sqcap) \text{ is a finite and complete lattice}$.
    }
    \step{complete}{A nonempty finite lattice is a complete lattice.}
    \begin{proof}
        \pf\ Know property \cite{moller_statitc_nodate}.
    \end{proof}
    \step{lattice}{$(p_X(S), \sqsubseteq, \sqcup, \sqcap)$ is a lattice.}
    \begin{proof}
        \step{poset}{$p_X(S)$ is a poset}
        \begin{proof}
            \step{poset-subset}{The subset of a poset is a poset under the same ordering relation.}
            \begin{proof}
                \pf\ Know property.\todo{source missing}
            \end{proof}
            \step{partition-subset}{$p_X(S) \subset S$}
            \begin{proof}
                \pf\ By the definition of a partition and lattices.
            \end{proof}
            \qedstep
            \begin{proof}
                \pf\ By \stepref{poset-subset} and \stepref{partition-subset}.
            \end{proof}
        \end{proof}
        \step{capcup}{$\forall x, y \in p_X(S) : x \sqcup y \in p_X(S) \land x \sqcap y \in p_X(S)$}
        \begin{proof}
            \pf\ By the definition of $p_X(S)$.
        \end{proof}
        \qedstep
        \begin{proof}
            \pf\ By \stepref{poset} and \stepref{capcup}.
        \end{proof}
    \end{proof}
    \step{finite}{$p_X(S)$ is finite.}
    \begin{proof}
        \pf\ By \autoref{lem:finite-partition}.
    \end{proof}
    \qedstep
    \begin{proof}
        \pf\ By \stepref{complete}, \stepref{lattice} and \stepref{finite}.
    \end{proof}
\end{proof}
