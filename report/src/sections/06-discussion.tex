\section{Discussion}\label{sec:discussion}
This section will discuss the results of the analysis and the implications of the results.

\begin{itemize}
    \item Summary of results
    \item Problems and limitations of the work
\end{itemize}


%Cortesi and Halder is hard to read, almost as if they make it harder to understand than needed. Their way cannot terminate.
%SQL was not they best use, as it has many implementation features that are different for platforms, therefore we might want to have used a more abstract language.

%The semantics for our language is not well defined, making our soundness proof a little bit shaky. Such as the boolean expressions, as they are not defined how they interact with lists or other boolean (Both our paper and cortesi and halder).
%There is no actual type system, so we assume that a variable has a type without formally defining it. 

%We use postgres SQL, but we have only defined a subset of the operations.
%Our language is more limited than Cortesi and Halder, as we do not have the same amount of operations.
%There are some programs in cortesi and halder that we cannot check, which are programs where more than one column is selected, due to the cover lattice only being able to handle one variable and not a column of variables. 
%Null values are not handled in our system, as we do not have a way to check for them. Meaning that we cannot check for null values in our system.
\subsection{Summary of results}\label{subsec:summaryresults}
Throughout the work progress, it was originally planned to have a working system with the analysis tool.
However,
due to time constraints and the complexity of the research,
it was deemed necessary to move the system to future work.
Due to this, there are no test results to discuss.
However, it is still possible to discuss the implications of the results that could have been found.


\subsection{Problems in the work}\label{subsec:workproblems}
Providing a framework for doing analysis on a database system proved harder than first expected.
This became clear in the choice of working with the paper~\cite{halder_abstract_2012} as a foundation to our work.
Although the paper provided an extensive syntax and semantics, it was not capable of being used as an analysis tool.
Altering it to be so would prove to be challenging for the following reasons.

We found that using their semantics for an analysis could result in it being non-terminating,
therefore deeming it not viable.
The non-termination can be seen in the way that they represent abstract tables.
Specifically, in \autoref{tab:table8h&c},
we clearly see how their semantics introduce duplicate entries in a table, where $\{Age^\#,Dno^\#,Sal^\#\}=\{{[}25,59{]},2,{[}1500,2499{]}\}$ appears twice.
\begin{table}[]
    \centering
    \caption{Table 8 from~\cite{halder_abstract_2012}}
    \begin{tabular}{lll}
        \toprule
        $Age^\#$ & $Dno^\#$ & $Sal^\#$ \\ \midrule
        {[}25,59{]}             & 2                       & {[}1500,2499{]}         \\
        {[}12,24{]}             & 1                       & {[}1500,2499{]}         \\
        {[}25,59{]}             & 2                       & {[}1500,2499{]}         \\
        {[}5,11{]}              & 1                       & {[}1500,2499{]}         \\
        {[}25,59{]}             & 3                       & {[}2500,10000{]}        \\
        {[}60,100{]}            & 1                       & {[}1500,2499{]}         \\
        {[}12,24{]}             & 2                       & {[}2500,10000{]}        \\ \bottomrule
    \end{tabular}\label{tab:table8h&c}
\end{table}
This means that a situation can happen, where adding more of the same tuples could end up enlarging the analysis indefinitely.

This is exactly what this paper tries to escape.
As mentioned in \autoref{subsubsec:abstract_domain_of_tables}, using abstract bags of abstract tuples eliminates the possibility of having duplicate tuple, ensuring a finite state space and terminating analysis.

Changing the way in which the abstractions are represented means that quite a lot of change is needed for the semantics.
So even though this paper provides a model that can be used for actual analysis, it also has its compromises.
Due to time constraints, it has not been possible to cover as much of SQL as \autoref{subsubsec:abstract_domain_of_tables} does.
This limits the amount of use cases the tool could be used for, naturally leaving an expansion of the tool to future work.

As mentioned in \autoref{subsec:summaryresults}, a plan was also established to implement a system that made use of the tool developed in this paper.
Unfortunately, it was not possible to make such a system, given the amount of work was required to ensure a terminating analysis.
Considering the complicated nature of the analysis, which quickly gets tricky if carried out in hand, a system would be required to provide a meaningful use in the future.


Moreover, the choice of SQL, as the language for the implementation, added more complication than necessary.
The reason for this is that the language has many implementation features that are different from dialect to dialect.
Therefore, it might have been better to use a more abstract language, as relational algebra, to implement the system.
Especially since the contribution in this paper only defined a subset of the operations in PostgreSQL.