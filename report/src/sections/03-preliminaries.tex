
\section{preliminaries}\label{sec:preliminaries}

\subsection{Lattices}\label{sec:lattices}

\begin{definition}
    A \textit{partial order} $(S, \sqsubseteq)$ is set $S$ equipped with a binary relation $\sqsubseteq$ that is reflexive, transitive and anti-symmetric.
\end{definition}

For $X \sqsubseteq S$ and $y \in S$ we take
\begin{equation*}
    X \sqsubseteq y \iff \forall x \in X : x \sqsubseteq y,
\end{equation*}
and analogous for $y \sqsubseteq X$.

\begin{definition}
    A \textit{complete lattice} $(S, \sqsubseteq, \sqcup, \sqcap)$ is a partial order $(S, \sqsubseteq)$ in which for all $X \subseteq S:$ $\bigsqcup X$ and $\bigsqcap X$ are defined,
        \begin{equation*}
            X \sqsubseteq \bigsqcup X \land \forall y \in S : X \sqsubseteq y \implies \bigsqcup X \sqsubseteq y,
        \end{equation*}
        and
        \begin{equation*}
            \bigsqcap X \sqsubseteq X \land \forall y \in S : y \sqsubseteq X \implies y \sqsubseteq \bigsqcap X.
        \end{equation*}
\end{definition}

As a shorthand we take $x \sqcup y = \bigsqcup \{x, y\}$ and $x \sqcap y = \bigsqcap \{x, y\}$.

\begin{definition}
    A \textit{lattice} $(S, \sqsubseteq, \sqcup, \sqcap)$ is a partial order $(S, \sqsubseteq)$ in which for all $x,y \in S:$ $x \sqcup y$ and $x \sqcap y$ are defined,
        \begin{equation*}
            \{x, y\} \sqsubseteq x \sqcup y \land \forall z \in S : \{x, y\} \sqsubseteq z \implies x \sqcup y \sqsubseteq z,
        \end{equation*}
        and
        \begin{equation*}
            x \sqcap y \sqsubseteq \{x, y\} \land \forall z \in S : z \sqsubseteq \{x, y\} \implies z \sqsubseteq x \sqcap y.
        \end{equation*}
\end{definition}

\begin{theorem}\label{thm:kleene_finite}
    In a complete lattice $L$ with finite height, every monotone function $f : L \rightarrow L$ has a unique fixed point
    \begin{equation*}
        lfp(f) = \bigsqcup\{f^n(\perp) \mid n \in \mathbb{N}\}
    \end{equation*}.
\end{theorem}

\begin{theorem}\label{thm:kleene_scott}
    In a complete lattice $L$, every Scott-continuous function $f : L \rightarrow L$ has a unique fixed point $lfp(f)$.
\end{theorem}

\todo[inline]{Casper says:
    The two theorems above should have a source.}

\subsection{Abstract Interpretation}

\subsection{Regular languages}



