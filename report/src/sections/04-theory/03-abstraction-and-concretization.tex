\subsection{Abstraction and Concretization}\label{subsec:abstraction-and-concretization}

\begin{align}
    \concrete_1 &: \clattice{\mathcal{R}}{REG} \rightarrow 2^{\Sigma^\star} \\
    \concrete_1(R) &= \mathcal{L}(R)
\end{align}

\begin{align}
    \concrete_2 &: \mathbb{T}^\# \rightarrow 2^\mathbb{T} \\
    \concrete_2 (e_1^\#, e_2^\#, \dots, e_n^\#) &= \concrete_1(e_1^\#) \times \concrete_1(e_2^\#) \times \dots \times \concrete_1(e_n^\#)
\end{align}

\begin{align}
    \concrete_3 &: (\mathbb{T}^\# \rightarrow \mathbb{N}^\#) \rightarrow 2^{(\mathbb{T} \rightarrow \mathbb{N})} \\
    \concrete_3 (t^\#) &= \left\{ t \in (\mathbb{T} \rightarrow \mathbb{N}) \left|\; \forall e^\# \in \mathbb{T}^\# : \left( \sum_{e \in \concrete_2(e^\#)} t(e) \right) \in \concrete_2(t^\#(e^\#))\right. \right\}
\end{align}

\begin{align}
    \concrete_4 &: \bigtimes_{i = 1}^{n} (\mathbb{T}^\#_i \rightarrow \mathbb{N}^\#_i) \rightarrow 2^{\bigtimes_{i = 1}^{n} (\mathbb{T}_i \rightarrow \mathbb{N}_i)} \\
    \concrete_4(d^\#) &= \concrete_4(t_1^\#, t_2^\#, \dots, t_n^\#) \\
                    &= \concrete_3(t_1^\#) \times \concrete_3(t_2^\#) \times \dots \times \concrete_3(t_n^\#)
\end{align}

\begin{align}
    \concrete_5(\env{t^\#}) &= \{ \env{t} \mid t \in \concrete_3(t^\#) \} \\
    \concrete_5(\env{d^\#}) &= \{ \env{d} \mid d \in \concrete_4(d^\#) \}
\end{align}

\begin{restatable}{theorem}{galois}\label{thm:galios}
    There exist an $\aabstract_5$ such that $\aabstract_5$ and $\concrete_5$ form a Galois connection.
\end{restatable}
