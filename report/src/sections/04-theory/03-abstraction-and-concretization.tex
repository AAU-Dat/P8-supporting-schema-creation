\subsection{Abstraction and Concretization}\label{subsec:abstraction-and-concretization}

\begin{align}
    \concrete_4 &: \clattice{\mathcal{R}}{REG} \rightarrow 2^{\Sigma^\star} \\
    \concrete_4(R) &= \mathcal{L}(R)
\end{align}

\begin{align}
    \concrete_3 &: \mathbb{T}^\# \rightarrow 2^\mathbb{T} \\
    \concrete_3 (e_1^\#, e_2^\#, \dots, e_n^\#) &= \concrete_4(e_1^\#) \times \concrete_4(e_2^\#) \times \dots \times \concrete_4(e_n^\#)
\end{align}

\begin{align}
    \concrete_2 &: (\mathbb{T}^\# \rightarrow \mathbb{N}^\#) \rightarrow 2^{(\mathbb{T} \rightarrow \mathbb{N})} \\
    \concrete_2(t^\#) &= \{t \in (\mathbb{T} \rightarrow \mathbb{N}) \mid \forall e \in \mathbb{T} : \forall e^\# \in \mathbb{T}^\# : e \in \concrete(e^\#) \implies t(e) \in \concrete(t^\#(e^\#))\}
\end{align}

\begin{align}
    \concrete_2 &: (\mathbb{T}^\# \rightarrow \mathbb{N}^\#) \rightarrow 2^{(\mathbb{T} \rightarrow \mathbb{N})} \\
    \concrete_2 (t^\#) &= \left\{ t \in (\mathbb{T} \rightarrow \mathbb{N}) \left|\; \forall e^\# \in \mathbb{T}^\# : \exists n \in \concrete_3(t^\#(e^\#)) : n = \sum_{e \in \concrete_3(e^\#)} t(e) \right. \right\}
\end{align}

\begin{align}
    \concrete(\env{t^\#}) = \{ \env{t} \mid t \in \concrete(t^\#) \}
\end{align}

\begin{align}
    \concrete_1 &: \bigtimes_{i = 1}^{n} (\mathbb{T}^\#_i \rightarrow \mathbb{N}^\#_i) \rightarrow 2^{\bigtimes_{i = 1}^{n} (\mathbb{T}_i \rightarrow \mathbb{N}_i)} \\
    \concrete_1(d^\#) &= \concrete_1(t_1^\#, t_2^\#, \dots, t_n^\#) \\
                    &= \concrete_2(t_1^\#) \times \concrete_2(t_2^\#) \times \dots \times \concrete_2(t_n^\#)
\end{align}

\begin{align}
    \concrete(\env{d^\#}) &= \{ \env{d} \mid d \in \concrete(d^\#) \}
\end{align}

\begin{restatable}{theorem}{galois}\label{thm:galios}
    There exist an $\aabstract_1$ such that $\aabstract_1$ and $\concrete_1$ form a Galois connection.
\end{restatable}
