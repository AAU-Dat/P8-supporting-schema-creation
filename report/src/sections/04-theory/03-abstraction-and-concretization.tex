\subsection{Abstraction and Concretization}\label{subsec:abstraction-and-concretization}

\todo[inline]{Casper says:
    $\concrete_1$ describes the correspondence between regular expressions and the strings that they represent, naturally they represent their respective language.
    The corresponding function for linear inequalities is missing, but should go here ones done.
}

\begin{align}
    \concrete_1 &: \clattice{\mathcal{R}}{REG} \rightarrow 2^{\Sigma^\star} \\
    \concrete_1(R) &= \mathcal{L}(R)
\end{align}

\todo[inline]{Casper says:
    $\concrete_2$ describes the correspondence between a tuple $e^\# \in \mathbb{T}^\#$, where $\mathbb{T}$ is the domain of some table and $\mathbb{T}^\#$ is it's abstract counterpart.
    An abstract tuple is thus mapped to the product of it's parts.
}

\begin{align}
    \concrete_2 &: \mathbb{T}^\# \rightarrow 2^\mathbb{T} \\
    \concrete_2 (e_1^\#, e_2^\#, \dots, e_n^\#) &= \concrete_1(e_1^\#) \times \concrete_1(e_2^\#) \times \dots \times \concrete_1(e_n^\#)
\end{align}

\todo[inline]{Casper says:
    An abstract table $t \in (\mathbb{T}^\# \rightarrow \mathbb{N}^\#)$ is mapped to it's corresponding table.
    As an example the very simple table $t^\#(a^\star b) = [1, 5]$, $t^\#(\overline{a^\star b}) = [0, 0]$, $t^\#(\top) = [1, 5]$,  $t^\#(\bot) = ?$, the concretisation $\concrete_3(t^\#) = T$, would contain $t$ where $t("aab") = 5$ and $t(s) = 0$ for all other strings $s$, but not $t'$ where $t'("aab") = 5$, $t("b") = 1$ and $t(s) = 0$ for all other strings $s$.

}

\begin{align}
    \concrete_3 &: (\mathbb{T}^\# \rightarrow \mathbb{N}^\#) \rightarrow 2^{(\mathbb{T} \rightarrow \mathbb{N})} \\
    \concrete_3 (t^\#) &= \left\{ t \in (\mathbb{T} \rightarrow \mathbb{N}) \left|\; \forall e^\# \in \mathbb{T}^\# : \left( \sum_{e \in \concrete_2(e^\#)} t(e) \right) \in \concrete_2(t^\#(e^\#))\right. \right\}
\end{align}

\todo[inline]{Casper says:
    Same as $\concrete_2$, but on a table to table basis.
}

\begin{align}
    \concrete_4 &: \bigtimes_{i = 1}^{n} (\mathbb{T}^\#_i \rightarrow \mathbb{N}^\#_i) \rightarrow 2^{\bigtimes_{i = 1}^{n} (\mathbb{T}_i \rightarrow \mathbb{N}_i)} \\
    \concrete_4(d^\#) &= \concrete_4(t_1^\#, t_2^\#, \dots, t_n^\#) \\
                    &= \concrete_3(t_1^\#) \times \concrete_3(t_2^\#) \times \dots \times \concrete_3(t_n^\#)
\end{align}

\todo[inline]{Casper says:
    As $\env{t}$ is a function that maps an attribute to it's corresponding row in a table concretizing it corresponds to concretizing the underlying table.
    Analogously $\env{d}$ is a function that maps table names to their respective tables, so again we concertize the all the underlying tables.
}

\begin{align}
    \concrete_5(\env{t^\#}) &= \{ \env{t} \mid t \in \concrete_3(t^\#) \} \\
    \concrete_5(\env{d^\#}) &= \{ \env{d} \mid d \in \concrete_4(d^\#) \}
\end{align}

\begin{restatable}{theorem}{galois}\label{thm:galios}
    There exist an $\aabstract_5$ such that $\aabstract_5$ and $\concrete_5$ form a Galois connection.
\end{restatable}
