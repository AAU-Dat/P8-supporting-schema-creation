\subsection{Abstract Syntax} \label{subsec:abstract-syntax}
% This section will introduce the syntactic sets and abstract syntax of SQAAL.
% The syntactic sets of SQAAL encompass the various elements and constructs that constitute the language's grammar.
% These sets define the valid combinations of symbols and keywords that form syntactically correct statements in SQAAL. They include components such as keywords, identifiers, operators, literals, and punctuation marks, each serving a specific role in expressing queries and analytical operations.

This paper adopts a modified version of the abstract syntax originally presented in~\cite{cortesi_abstract_2013}.
Omitted parts within this simplified abstract syntax correspond to features or constructs that are not utilized or discussed in the context of the current application.
We give a brief outline of changes:
\begin{itemize}
    \item The equivalent of database variables (attributes) are now denoted $v_t$ and are called column variables or attributes.
    \item Table names, denoted $v_d$ are now explicit and are required to be present in SQL commands.
    \item The non deterministic choice operator $\texttt{[]}$ between two instructions has been introduced.
    \item In $v_a \coloneq \; ?$ instructions we allow the $?$ to be further specified, allowing abstract domains to take its place.
\end{itemize}

Further we differentiate between the instructions allowed on the edges of a program graph $\mathbb{I}$, and the instructions allowed in the language for which the analysis is allowed $\mathbb{P}$.
We will later in this section formalize the conversion between program and program graph, but before we can do that we need to introduce the syntax.

The abstract syntax of SQAAL is defined in terms of the syntactic sets introduced in \autoref{fig:syntatic-sets}.

\begin{figure}[htb!]
    \center
    \begin{tabular}{r p{0.6\columnwidth }}
        $n \in \mathbb{Z}$                                  & Integers                                                  \\
        $k \in \mathbb{S}$                                  & Strings                                                   \\
        $\mathscr{I} \in \mathbf{INT}$                      & Interval unions                                           \\
        $R \in \mathbf{REG}$                                & Regular expressions                                       \\
        $c \in \mathbb{C}$                                  & Constants                                                 \\
        $v_a \in \mathbb{V}_a$                              & Application variables                                     \\
        $v_d \in \mathbb{V}_d$                              & Database variables                                        \\
        $v_t \in \mathbb{V}_t$                              & Column variables                                          \\
        $v \in \mathbb{V} = \mathbb{V}_a \cup \mathbb{V}_t$ & Variables                                                 \\
        $e \in \mathbb{E}$                                  & Arithmetic expressions                                    \\
        $b \in \mathbb{B}$                                  & Boolean expressions                                       \\
        $A_{sql} \in \mathbb{A}_{sql}$                      & Action part of SQL commands                               \\
        $C_{sql} \in \mathbb{C}_{sql}$                      & SQL commands                                              \\
        $I \in \mathbb{I}$                                  & Program graph instructions                                \\
        $P \in \mathbb{P}$                                  & Program instructions                                      \\
    \end{tabular}
    \caption{Syntatic Sets}
    \label{fig:syntatic-sets}
\end{figure}

The abstract syntax of SQAAL is presented in \autoref{fig:abstract-syntax}.
For simplicity's sake we assume that $\mathbb{V}_a$, $\mathbb{V}_d$ and $\mathbb{V}_t$ are mutually disjoint.
We allow non-deterministic values of $R$ and $\mathscr{I}$ to be constants for the purpose of defining the abstract semantics later on, however we do not allow such constant values to appear in expressions of programs in $\mathbb{P}$, however we do allow such values to appear on the right hand side of an assignment if they are not part of an expression.
Note as previously mentioned that bold characters denote vectors/list of values, thus $\mathbf{e}$ is a list of expressions and $\mathbf{v}_t$ a list of attributes.

\begin{figure}[htb!]
    \center
    \begin{tabular}{r p{0.65\columnwidth}}
        $? \Coloneqq$ & $R \mid \mathscr{I}$ \\
        $c \Coloneqq$ & $n \mid k \mid \; ?$ \\
        $v \Coloneqq$ & $v_t \mid v_a$ \\
        $e \Coloneqq$ & $c \mid v \mid op_u\; e \mid \;e_1 op_b\; e_2$ for $op_u = \{-, lower, upper, bit\_length, length\}$ and $op_b = \{+, -, *, /, ||\}$ \\

        $b \Coloneqq$ & $e_1 = e_2 \mid e_1 \leq e_2 \mid e_1 < e_2 \mid \neg b \mid b_1 \lor b_2 \mid b_1 \land b_2 \mid \texttt{true} \mid \texttt{false} \mid \forall v_n b \mid \exists v_n b$ \\
        $A_{sql} \Coloneqq$ & $select(v_a, v_d, \mathbf{v}_t) \mid update(v_d, \mathbf{v}_t, \mathbf{e}) \mid delete(v_d) \mid insert(v_d, \mathbf{v}_t, \mathbf{e})$ \\
        $C_{sql} \Coloneqq$ & $\langle A_{sql}, b \rangle $ \\
        $I \Coloneqq$ & $\texttt{skip} \mid v_a \coloneqq e \mid v_a \coloneqq \; ? \mid C_{sql} \mid b$ \\
        $P \Coloneqq$ & $\texttt{skip} \mid v_a \coloneqq e \mid v_a \coloneqq \; ? \mid C_{sql} \mid \texttt{if } b \texttt{ then } P_1 \texttt{ else } P_2 \mid \texttt{while } b \texttt{ do } P \mid P_1; P_2 \mid P_1 \texttt{[]} P_2 $ \\
    \end{tabular}
    \caption{Abstract Syntax}
    \label{fig:abstract-syntax}
\end{figure}

Now we are ready to define the conversion from program to program graph.
The conversion is given by the function $edges$ which in the recursive cases are defined as follows, starting with the if-then-else statement, as visualized in \autoref{fig:tikz-program-graph-if}:
\todo{we should ref all the equations, making sure it is clear for the reader}
\begin{equation}
    \begin{split}
        &edges(q_\circ \rightsquigarrow q_\bullet) \lBrack \texttt{if } b \texttt{ then } P_1 \texttt{ else } P_2 \rBrack = \\
        &\quad \text{let } q, q' \text{ be new nodes} \\
        &\quad\quad E_{b} = edges(q_\circ \rightsquigarrow q) \lBrack b \rBrack \\
        &\quad\quad E_{\neg b} = edges(q_\circ \rightsquigarrow q') \lBrack \neg b \rBrack \\
        &\quad\quad E_{P_1}= edges(q \rightsquigarrow q_\bullet) \lBrack P_1 \rBrack \\
        &\quad\quad E_{P_2}= edges(q' \rightsquigarrow q_\bullet) \lBrack P_2 \rBrack \\
        &\quad \text{in } E_{b} \cup E_{\neg b} \cup E_{P_1} \cup E_{P_2}
    \end{split}\label{eq:equation8}
\end{equation}

\begin{figure}[htb!]
    \center
    \begin{tikzpicture}
    \tikzstyle{arrow} = [thick,->,>=stealth]
    \node [circle, draw] (start) at (0,0){};
    \node [circle, draw, fill=black] (end) at (0,-2.3){};
    \node (a) [cloud,  cloud puffs=9.4, minimum width=1cm, draw, right=2cm of start, shift={($(start.south)+(0.2,-1cm)$)}] {$I_2$};
    \node (b) [cloud,  cloud puffs=9.4, minimum width=1cm, draw, left=2cm of start, shift={($(start.south)+(-0.2,-1cm)$)}] {$I_1$};

    \draw[arrow, ->] (start) to node[below,xshift=6, yshift=10] {$\neg b$} (a);
    \draw[arrow, ->] (start) to node[below,xshift=-4, yshift=10] {$b$} (b);
    \draw[arrow, ->] (a) to node[below,scale=.7] {} (end);
    \draw[arrow, ->] (b) to node[below,scale=.7] {} (end);
\end{tikzpicture}
    \caption{Program graph of an if-else-statement}
    \label{fig:tikz-program-graph-if}
\end{figure}

And for the while statement, as visualized in \autoref{fig:tikz-program-graph-loop}:

\begin{equation}
    \begin{split}
        &edges(q_\circ \rightsquigarrow q_\bullet) \lBrack \texttt{while } b \texttt{ do } P \rBrack = \\
        &\quad \text{let } q \text{ be a new node} \\
        &\quad\quad E_{b} = edges(q_\circ \rightsquigarrow q) \lBrack b \rBrack \\
        &\quad\quad E_{\neg b} = edges(q_\circ \rightsquigarrow q_\bullet) \lBrack \neg b \rBrack \\
        &\quad\quad E_{P} = edges(q \rightsquigarrow q_\circ) \lBrack P \rBrack \\
        &\quad \text{in } E_{b} \cup E_{\neg b} \cup E_{P}
    \end{split}\label{eq:equation9}
\end{equation}

\begin{figure}[htb!]
    \center
    \begin{tikzpicture}
    \tikzstyle{arrow} = [thick,->,>=stealth]
    \node (a) [cloud,  cloud puffs=9.4, minimum width=1cm, draw] {S};
    \node [circle, draw, above=of a] (start) {};
    \node [circle, draw, fill=black, below=of a] (end) {};

    \draw[arrow, ->] ([shift={(0.0,0.045)}]a.east) to[bend right] node[below,scale=.7] {} (start);
    \draw[arrow, ->] (start) -- node[right,scale=.70,xshift=-15]{b} (a);
    \draw[arrow, ->] (start) to[bend right=55] node[right,scale=.70,xshift=-20]{$\neg b$} (end);
\end{tikzpicture}
    \caption{Program graph of a while-do-statement}
    \label{fig:tikz-program-graph-loop}
\end{figure}

And for the composition statement, as visualized in \autoref{fig:tikz-composition-graph}:

\begin{equation}
    \begin{split}
        &edges(q_\circ \rightsquigarrow q_\bullet) \lBrack P_1; P_2 \rBrack = \\
        &\quad \text{let } q \text{ be a new node} \\
        &\quad\quad E_{P_1} = edges(q_\circ \rightsquigarrow q) \lBrack P_1 \rBrack \\
        &\quad\quad E_{P_2} = edges(q \rightsquigarrow q_\bullet) \lBrack P_2 \rBrack \\
        &\quad \text{in } E_{P_1} \cup E_{P_2}
    \end{split}\label{eq:equation10}
\end{equation}


\begin{figure}
    \center
    \begin{tikzpicture}
    \tikzstyle{arrow} = [thick,->,>=stealth]
    \node [circle, draw] (start) at (0,0){};
    \node (a) [cloud,  cloud puffs=9.4, minimum width=1cm, draw, shift={($(start.east)+(1.5,0)$)}] {$P_1$};
    \node [circle, draw, shift={($(a.east)+(1.5,0)$)}] (b) {};
    \node (c) [cloud,  cloud puffs=9.4, minimum width=1cm, draw, shift={($(b.east)+(1.5,0)$)}] {$P_2$};
    \node [circle, draw, shift={($(c.east)+(1.5,0)$)}, fill=black] (end) {};

    \draw[arrow, ->] (start) to node[above,scale=.7] {} (a);
    \draw[arrow, ->] (a) to node[above,scale=.7] {} (b);
    \draw[arrow, ->] (b) to node[above,scale=.7] {} (c);
    \draw[arrow, ->] (c) to node[above,scale=.7] {} (end);
\end{tikzpicture}

    \caption{Program graph of two instructions in sequence}
    \label{fig:tikz-composition-graph}
\end{figure}


And for non-deterministic choice statement, as visualized in \autoref{fig:non-deterministic-choice}:


\begin{equation}
    \begin{split}
        &edges(q_\circ \rightsquigarrow q_\bullet) \lBrack P_1 \texttt{[]} P_2 \rBrack = \\
        &\quad \text{let } q, q' \text{ be new nodes} \\
        &\quad\quad E_{q} = edges(q_\circ \rightsquigarrow q) \lBrack \texttt{skip} \rBrack \\
        &\quad\quad E_{q'} = edges(q_\circ \rightsquigarrow q') \lBrack \texttt{skip} \rBrack \\
        &\quad\quad E_{P_1}= edges(q \rightsquigarrow q_\bullet) \lBrack P_1 \rBrack \\
        &\quad\quad E_{P_2}= edges(q' \rightsquigarrow q_\bullet) \lBrack P_2 \rBrack \\
        &\quad \text{in } E_{q} \cup E_{q'} \cup E_{P_1} \cup E_{P_2}
    \end{split}\label{eq:equation13}
\end{equation}

\begin{figure}[htb!]
    \center
    \begin{tikzpicture}
    \tikzstyle{arrow} = [thick,->,>=stealth]
    \node [circle, draw] (start) at (0,0){};
    \node [circle, draw, fill=black] (end) at (0,-2.3){};
    \node (a) [cloud,  cloud puffs=9.4, minimum width=1cm, draw, right=2cm of start, shift={($(start.south)+(0.2,-1cm)$)}] {$P_2$};
    \node (b) [cloud,  cloud puffs=9.4, minimum width=1cm, draw, left=2cm of start, shift={($(start.south)+(-0.2,-1cm)$)}] {$P_1$};

    \draw[arrow, ->] (start) to node[below,xshift=12, yshift=10] {$\texttt{skip}$} (a);
    \draw[arrow, ->] (start) to node[below,xshift=-12, yshift=10] {$\texttt{skip}$} (b);
    \draw[arrow, ->] (a) to node[below,scale=.7] {} (end);
    \draw[arrow, ->] (b) to node[below,scale=.7] {} (end);
\end{tikzpicture}

    \caption{Program graph of a non-deterministic choice}
    \label{fig:non-deterministic-choice}
\end{figure}

And in the base case of $I$:

\begin{equation*}
    edges(q_\circ \rightsquigarrow q_\bullet) \lBrack I \rBrack = \{q_\circ \xrightarrow{I} q_\bullet\}\label{eq:equation11}
\end{equation*}

Thus, for some program $P$, the program graph of $P$ is simply $edges(q_\whitepointerright \rightsquigarrow q_\blackpointerleft)\lBrack P \rBrack$.
