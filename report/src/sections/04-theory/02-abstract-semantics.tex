\subsection{Abstract Semantics}

Let $S(v)$ the function that in beds $v$ in a singleton table.

% todo Casper says: We don't need to prove this... I think. Remove once confirmed

% \begin{restatable}{lemma}{deletemonotone}
%     $\abssem{C_{select}^\#}$ is monotone.
% \end{restatable}
% \pfsketch{
%     $\abssem{C_{select}^\#}$ leaves the database unchanged which is clearly monotone.
% }

\begin{align*}
    S^\# \llbracket C_{insert}^\# \rrbracket (\rho_{t^\#}, \rho_{a^\#})
    &= S^\# \llbracket \langle insert^\#(\overset{\rightarrow}{v_d^\#}, \overset{\rightarrow}{e^\#}), \phi^\# \rangle \rrbracket (\rho_{t^\#}, \rho_{a^\#}) \\
    &= (\rho_{t_1^\#}, \rho_{a^\#}) \\
    \text{where } target(C_{insert}^\#) &= t^\# \\
    \overset{\rightarrow}{v_d^\#} &= \langle a_1^\#, a_2^\#, \dots, a_n^\# \rangle \\
    &= attr(t^\#) \\
    E^\# \llbracket \overset{\rightarrow}{e^\#} \rrbracket (\rho_{t^\#}, \rho_{a^\#})
    &= \overset{\rightarrow}{r^\#} \\
    &= \langle r_1^\#, r_2^\#, \dots, r_n^\# \rangle \\
    l_{new}^\# &= \langle r_1^\#/a_1^\#, r_2^\#/a_2^\#, \dots r_n^\#/a_n^\# \rangle \\
    \rho_{t_1^\#}(\overset{\rightarrow}{x^\#}) &= \rho_{\textcolor{red}{t^\# \sqcup S(l_{new}^\#)}}(\overset{\rightarrow}{x^\#})
\end{align*}

% todo same as above

% \begin{restatable}{lemma}{insertmonotone}
%     $S^\# \llbracket C_{insert}^\# \rrbracket$ is monotone.
% \end{restatable}
% \pfsketch{
%     Regardless of $(\env{\abstable}, \env{\absvars})$ the same element is inserted, thus the ordering between any two environments after $S^\# \llbracket C_{insert}^\# \rrbracket$ is applied is the same, because the ordering is determined by the elements that were already in the environments.
% }

\begin{align*}
    \abssem{C^\#_{update}}(\env{\abstable}, \env{\absvars})
    &= \abssem{\langle update^\#(\absattrs, \absexps), \abspred \rangle} (\env{\abstable}, \env{\absvars}) \\
    &= (\env{\abstable_1}, \env{\absvars}) \\
    \text{where } \env{\abstable_1}(\absattr) &=
    \begin{cases}
        \env{\abstable \downarrow_t \abspred}(\absattr) \textcolor{red}{\sqcup} \env{\abstable \downarrow_u \abspred}(\absattr) \textcolor{red}{\sqcup} \env{\abstable \downarrow_f \abspred}(\absattr) \\
        \text{if } \absattr \notin \absattrs \\
        \absexpsem{\absexp{i}}(\env{\abstable \downarrow_t \abspred}, \env{\absvars}) \textcolor{red}{\sqcup}
        \left( \absexpsem{\absexp{i}}(\env{\abstable \downarrow_u \abspred}, \env{\absvars}) \sqcup \env{\abstable \downarrow_u \abspred}(\absattr) \right) \textcolor{red}{\sqcup} \env{\abstable \downarrow_f \abspred}(\absattr) \\
        \text{if $\absattr$ is the $i$th component of $\absattrs$ and $\absexp{i}$ is the $i$th componenet of $\absexps$} \\
    \end{cases}
\end{align*}

% todo same as above

% \begin{restatable}{lemma}{updatemonotone}
%     $\abssem{C_{update}^\#}$ is monotone.
% \end{restatable}
% \pfsketch{
%     For $\abstable \sqsubseteq \abstablep$ if $\abssem{C_{update}^\#}$ updates elements in $\abstable$ the same elements must be updated in $\abstablep$, therefore the order is determined by the remaining unchanged elements and therefore order is preserved.
%     If an element not in $\abstable$ is update is changed in $\abstablep$ order is still preserved as the element is 'swapped' for another.
% }

\begin{align*}
    \abssem{C^\#_{delete}}(\env{\abstable}, \env{\absvars})
    &= \abssem{\langle delete^\#(\absattrs), \abspred \rangle} (\env{\abstable}, \env{\absvars}) \\
    &= (\env{\abstable_1}, \env{\absvars}) \\
    \text{where } \env{\abstable_1}(\absattr) &= \env{\abstable \downarrow_u \abspred}(\absattr) \textcolor{red}{\sqcup} \env{\abstable \downarrow_f \abspred}(\absattr)
\end{align*}

% todo same as above

% \begin{restatable}{lemma}{deletemonotone}
%     $\abssem{C_{delete}^\#}$ is monotone.
% \end{restatable}
% \pfsketch{
%     For $\abstable \sqsubseteq \abstablep$ if $\abssem{C_{delete}^\#}$ deletes elements in $\abstable$ the same elements in must also be deleted in $\abstablep$ and order is preserved, further deleting elements only in $\abstablep$ will 'close the gap' between $\abstable$ and $\abstablep$ but the order is still preserved.
% }

