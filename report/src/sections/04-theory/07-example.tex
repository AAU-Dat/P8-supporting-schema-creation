\section{Example}\label{sec:example}
In this section we will present an example of a program that could be analyzed using the abstract interpretation presented in this paper.

The program in Figure\ref{fig:program-code} is a simple program that simulates a bank system With delete, input, select, and update operations.
\begin{figure}
    \begin{minted}[frame=lines, linenos, escapeinside=||, mathescape=true]{text}
while true do(
    vName := [A-Z][a-z]* [A-Z][a-z]*;
    vBalance := |$\top$|;
    (Insert(account, <Name, Balance>, <vName, vBalance>), true)
)[](
    vName := |$\top$|;
    (delete(account), name = vName)
)[](
    vName := [A-Z][a-z]* [A-Z][a-z]*;
    toName := [A-Z][a-z]* [A-Z][a-z]*;
    amount := [0,+|$\infty$|];
    (select (vBalance, account, <name>), name = vName)
    (select (toBalance, account, <name>), name = toName)
    if(vBalance - amount) < 0
        then skip
    else(
        (update(account, <balance>, <vBalance - amount>), name = vName);
        (update(account, <balance>, <toBalance + amount>), name = toName);
    )
)
    \end{minted}
    \caption{Code for the program example}
    \label{fig:program-code}
\end{figure}

The program in Figure\ref{fig:program-code} can be broken up into the following set of instructions.

\begin{align} \label{eq:example-instructions}
    \begin{split}
    vName := [A-Z][a-z]* [A-Z][a-z]*;
    \end{split}
    & I_1\\
    \begin{split}
        vBalance := \top; 
    \end{split}
    & I_2\\
    \begin{split}
        &(Insert(account, <Name, Balance>, \\
        &<vName, vBalance>), true)
    \end{split} 
    & I_3\\
    \begin{split}
        vName := \top;
    \end{split} 
    & I_4\\
    \begin{split}
        (delete(account), name = vName)
    \end{split} 
    & I_5\\
    \begin{split}
        vName := [A-Z][a-z]* [A-Z][a-z]*;
    \end{split} 
    & I_6\\
    \begin{split}
        toName := [A-Z][a-z]* [A-Z][a-z]*;
    \end{split} 
    & I_7\\
    \begin{split}
        amount := [0,+ \infty];
    \end{split} 
    & I_8\\
    \begin{split}
        &(select (vBalance, account, <name>), \\
        &name = vName)
    \end{split} 
    & I_9\\
    \begin{split}
        &(select (toBalance, account, <name>), \\
        &name = toName)
    \end{split} 
    & I_{10}\\
    \begin{split}
        &(update(account, <balance>, \\
        &<vBalance - amount>), name = vName);
    \end{split} 
    & I_{11}\\
    \begin{split}
        &(update(account, <balance>, \\
        &<toBalance + vBalance>), name = toName);
    \end{split}
    & I_{12}
\end{align}

A program graph is then made on the basis of the program in Figure\ref{fig:program-code} using the abstract interpretation.

\todo[inline]{insert first program graph}


The program graph can then be simplified to the following program graph.

\todo[inline]{insert second program graph}

From the states of the program graph we get the following equations.

\begin{align}\label{eq:exaple-equation}
\begin{split}
    A(q_\whitepointerright)=A(q_\whitepointerright) &\cup \hat E_\whitepointerright \\&\cup \mathcal{\hat S} \lBrack skip \rBrack(A(q_{t4}))\\&\cup \mathcal{\hat S} \lBrack I_{12} \rBrack (A(q_{t6})) \\&\cup \mathcal{\hat S} \lBrack I_3 \rBrack (A(q_i)) \\&\cup \mathcal{\hat S} \lBrack I_5 \rBrack (A(q_d))
\end{split}\\
A(q_i)=A(q_i)&\cup \mathcal{\hat S} \lBrack I_1;I_2 \rBrack (A(q_\whitepointerright ))\\
A(q_d)=A(q_d)&\cup \mathcal{\hat S} \lBrack I_4 \rBrack (A(q_\whitepointerright ))\\
A(q_{t1})=A(q_{t1})&\cup \mathcal{\hat S} \lBrack I_6;I_7;I_8 \rBrack (A(q_\whitepointerright))\\
A(q_{t2})=A(q_{t2})&\cup \mathcal{\hat S} \lBrack I_9 \rBrack (A(q_{t1}))\\
A(q_{t3})=A(q_{t3})&\cup \mathcal{\hat S} \lBrack I_{10} \rBrack (A(q_{t2}))\\
A(q_{t4})=A(q_{t4})&\cup \mathcal{\hat S} \lBrack b \rBrack (A(q_{t3}))\\
A(q_{t5})=A(q_{t5})&\cup \mathcal{\hat S} \lBrack \neg b \rBrack (A(q_{t3}))\\
A(q_{t6})=A(q_{t6})&\cup \mathcal{\hat S} \lBrack I_{11} \rBrack (A(q_{t5}))\\
A(q_\blackpointerleft)=A(q_\blackpointerleft)
\end{align}

When solving the equations in\ref{eq:exaple-equation} using the semantics in this paper, we get the following fixpoint.

\begin{align}
    \label{eq:example-fixed-point-1}
    A(q_{\whitepointerright})=
    \begin{split}
        &\left\{\begin{matrix}
                   \left.\begin{matrix*}[l]
                             \texttt{vName}\\
                             \texttt{vBalance}\\
                             \texttt{toName}\\
                             \texttt{toBalance}\\
                             \texttt{amount}\\
                             \texttt{account}
                   \end{matrix*}\right|
                   \left.\begin{matrix}
                             \bot\\
                             \bot\\
                             \bot\\
                             \bot\\
                             \bot\\
                             \emptyset
                   \end{matrix}\right|
                   \left.\begin{matrix}
                             NR\\
                             \top\\
                             \bot\\
                             \bot\\
                             \bot\\
                             \mathsf{List} \; \{(NR,\top)\}
                   \end{matrix}\right|
                   \left.\begin{matrix}
                             \top\\
                             \bot\\
                             \bot\\
                             \bot\\
                             \bot\\
                             \emptyset
                   \end{matrix}\right|
        \end{matrix}\right.\\
        &\left.\begin{matrix}
                  \left.\begin{matrix}
                            NR\\
                            \mathsf{List} \; \{\top\}\\
                            NR\\
                            \mathsf{List} \; \{\top\}\\
                            +\\
                            \mathsf{List} \{(NR,\top)\}
                  \end{matrix}\right|
                  \left.\begin{matrix}
                            NR\\
                            \top\\
                            NR\\
                            \mathsf{List} \; \{\top\}\\
                            +\\
                            \mathsf{List} \{(NR,\top)\}
                  \end{matrix}\right|
                  \begin{matrix}
                      \top\\
                      \mathsf{List} \; \{\top\}\\
                      NR\\
                      \mathsf{List} \; \{\top\}\\
                      +\\
                      \mathsf{List} \; \{(NR,\top)\}
                  \end{matrix}
        \end{matrix}\right\}
    \end{split}
\end{align}

\begin{align}
    A(q_i)=\left\{\begin{matrix}
                      \left.\begin{matrix*}[l]
                                \texttt{vName}\\
                                \texttt{vBalance}\\
                                \texttt{toName}\\
                                \texttt{toBalance}\\
                                \texttt{amount}\\
                                \texttt{account}
                      \end{matrix*}\right|
                      \left.\begin{matrix}
                                NR\\
                                \top\\
                                \bot\\
                                \bot\\
                                \bot\\
                                \emptyset
                      \end{matrix}\right|
                      \left.\begin{matrix}
                                NR\\
                                \top\\
                                \bot\\
                                \bot\\
                                \bot\\
                                \mathsf{List} \; \{(NR,\top)\}
                      \end{matrix}\right|
                      \begin{matrix}
                          NR\\
                          \top\\
                          NR\\
                          \mathsf{List} \; \{\top\}\\
                          +\\
                          \mathsf{List} \; \{(NR,\top)\}
                      \end{matrix}
    \end{matrix}\right\}
\end{align}

\begin{align}
    A(q_d)=\left\{\begin{matrix}
                      \left.\begin{matrix}
                                \texttt{vName}\\
                                \texttt{vBalance}\\
                                \texttt{toName}\\
                                \texttt{toBalance}\\
                                \texttt{amount}\\
                                \texttt{account}
                      \end{matrix}\right|
                      \left.\begin{matrix}
                                \top\\
                                \bot\\
                                \bot\\
                                \bot\\
                                \bot\\
                                \emptyset
                      \end{matrix}\right|
                      \left.\begin{matrix}
                                \top\\
                                \top\\
                                \bot\\
                                \bot\\
                                \bot\\
                                \mathsf{List} \; \{(NR, \top)\}
                      \end{matrix}\right|
                      \begin{matrix}
                          \top\\
                          \mathsf{List} \; \{\top\}\\
                          NR\\
                          \mathsf{List} \; \{\top\}\\
                          +\\
                          \mathsf{List} \; \{(NR,\top)\}
                      \end{matrix}
    \end{matrix}\right.
\end{align}

\begin{align}
    A(q_{t_1})=
    \begin{split}
        &\left\{\begin{matrix}
                   \left.\begin{matrix*}[l]
                             \texttt{vName}\\
                             \texttt{vBalance}\\
                             \texttt{toName}\\
                             \texttt{toBalance}\\
                             \texttt{amount}\\
                             \texttt{account}
                   \end{matrix*}\right|
                   \left.\begin{matrix}
                             NR\\
                             \bot\\
                             NR\\
                             \bot\\
                             +\\
                             \emptyset
                   \end{matrix}\right|
                   \left.\begin{matrix}
                             NR\\
                             \top\\
                             NR\\
                             \bot\\
                             +\\
                             \mathsf{List} \; \{(NR, \top)\}
                   \end{matrix}\right|
        \end{matrix}\right. \\
        &\left.\begin{matrix}
                  \left.\begin{matrix}
                            NR\\
                            \mathsf{List} \; \{\top\}\\
                            NR\\
                            \mathsf{List} \; \{\top\}\\
                            +\\
                            \mathsf{List} \; \{(NR, \top)\}
                  \end{matrix}\right|
                  \begin{matrix}
                      NR\\
                      \top\\
                      NR\\
                      \mathsf{List} \; \{\top\}\\
                      +\\
                      \mathsf{List} \; \{(NR, \top)\}
                  \end{matrix}
        \end{matrix}\right\}
    \end{split}
\end{align}

\begin{align}
    A(q_{t_2})=\left\{\begin{matrix}
                          \left.\begin{matrix*}[l]
                                    \texttt{vName}\\
                                    \texttt{vBalance}\\
                                    \texttt{toName}\\
                                    \texttt{toBalance}\\
                                    \texttt{amount}\\
                                    \texttt{account}
                          \end{matrix*}\right|
                          \left.\begin{matrix}
                                    NR\\
                                    \emptyset\\
                                    NR\\
                                    \bot\\
                                    +\\
                                    \emptyset
                          \end{matrix}\right|
                          \left.\begin{matrix}
                                    NR\\
                                    \mathsf{List} \{\top\}\\
                                    NR\\
                                    \bot\\
                                    +\\
                                    \mathsf{List} \; \{(NR, \top)\}
                          \end{matrix}\right|
                          \begin{matrix}
                              NR\\
                              \mathsf{List} \{\top\}\\
                              NR\\
                              \mathsf{List} \{\top\}\\
                              +\\
                              \mathsf{List} \; \{(NR, \top)\}
                          \end{matrix}
    \end{matrix}\right\}
\end{align}

\begin{align}
    A(q_{t_3})=\left\{\begin{matrix}
                          \left.\begin{matrix*}[l]
                                    \texttt{vName}\\
                                    \texttt{vBalance}\\
                                    \texttt{toName}\\
                                    \texttt{toBalance}\\
                                    \texttt{amount}\\
                                    \texttt{account}
                          \end{matrix*}\right|
                          \left.\begin{matrix}
                                    NR\\
                                    \emptyset\\
                                    NR\\
                                    \emptyset\\
                                    +\\
                                    \emptyset
                          \end{matrix}\right|
                          \begin{matrix}
                              NR\\
                              \mathsf{List} \{\top\}\\
                              NR\\
                              \mathsf{List} \{\top\}\\
                              +\\
                              \mathsf{List} \; \{(NR, \top)\}
                          \end{matrix}
    \end{matrix}\right\}
\end{align}

\begin{align}
    A(q_{t_4})=\left\{\begin{matrix}
                          \left.\begin{matrix*}[l]
                                    \texttt{vName}\\
                                    \texttt{vBalance}\\
                                    \texttt{toName}\\
                                    \texttt{toBalance}\\
                                    \texttt{amount}\\
                                    \texttt{account}
                          \end{matrix*}\right|
                          \begin{matrix}
                              NR\\
                              \mathsf{List} \{\top\}\\
                              NR\\
                              \mathsf{List} \{\top\}\\
                              +\\
                              \mathsf{List} \; \{(NR, \top)\}
                          \end{matrix}
    \end{matrix}\right\}
\end{align}

\begin{align}
    \label{eq:example-fixed-point-7}
    A(q_{t_5})=\left\{\begin{matrix}
                          \left.\begin{matrix*}[l]
                                    \texttt{vName}\\
                                    \texttt{vBalance}\\
                                    \texttt{toName}\\
                                    \texttt{toBalance}\\
                                    \texttt{amount}\\
                                    \texttt{account}
                          \end{matrix*}\right|
                          \begin{matrix}
                              NR\\
                              \mathsf{List} \{\top\}\\
                              NR\\
                              \mathsf{List} \{\top\}\\
                              +\\
                              \mathsf{List} \; \{(NR, \top)\}
                          \end{matrix}
    \end{matrix}\right\}
\end{align}

\begin{align} 
\label{eq:example-fixed-point-8}
    A(q_{\blackpointerleft})=\emptyset
\end{align}


You would then be able to use the fixpoint\ref{eq:exaple-equation} to analyze the program and check for some requested properties.
In this case one of the properties could be to check if it is possible to have accounts with a negative balance, which in this case would be possible.
Or you could check if it is possible to have accounts with names that are not in the format of the regular expression $[A-Z][a-z]* [A-Z][a-z]*$, which in this case would hold.