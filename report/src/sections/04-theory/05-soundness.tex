\subsection{Soundness}\label{subsec:soundness}

In this subsection we give an informal argument of soundness for the specification of our analysis.
A formal proof would be far too long and should ideally be verified with a formal proof management system.
A draft of the concrete semantics is given in \autoref{sec:concrete-semantics}.

For the proofs and proof-sketches in this section we allow ourselves to substitute terms $s \into \lookupcl(v)$ with $s$, the justification is that $s \into \lookupcl(v)$ will only ever overestimate $s$, that is $s \subseteq s \into \lookupcl(v)$.
More specifically, if one where to take $\ab{\mathcal{S}}'\lBrack\cdot\rBrack$ to be $\abssem{\cdot}$ where such a substitution had taken place, we conjecture that you would find that $\concrete_1 \left( \ab{\mathcal{S}}'\lBrack I \rBrack (\ab{\environment})\right) \subseteq \concrete_1 \left( \abssem{I}(\ab{\environment}) \right)$ for all $I$ and $\ab{\environment}$, and therefore if $\ab{\mathcal{S}}'\lBrack \cdot \rBrack$ was shown to be sound then $\abssem{\cdot}$ would also be sound .

\begin{conjecture}\label{thm:sound-exp}
    $\absexpsem{\cdot}$ is sound, formally
    \begin{equation*}
        \environment \in \concrete_2(\ab{\environment}) \implies \expsem{e}(\environment) \in \concrete_{4a}(\absexpsem{e}(\ab{\environment}))
    \end{equation*}
\end{conjecture}

\begin{proof}
    \pfsketch\
    The proof would proceed by structural induction on $\mathbb{E}$.
    It should be clear that the base, when the expression is a single constant, is true.
    The inductive case would rely on the inductive hypothesis, and the fact that each $\aab{op_u}$ and $\aab{op_b}$ are overestimates of their respective operations and the fact that the operation is done between each element in the case where a column of values is suspected.
    When we say overestimates, what we precisely mean is that if $s_i \in \concrete_6(\ab{s_i})$ for $i \in \{1, 2\}$ then $s_1 \op s_2 \in \concrete_6(\ab{s_1} \; \aab{op_b} \; \ab{s_2})$.
    Or if $s \in \concrete_6(\ab{s})$ then $\op s \in \concrete_6(\aab{op_u} \; \ab{s})$.
\end{proof}


\begin{conjecture}\label{thm:sound-bool}
    $\absboolsem{\cdot}$ is sound, formally
    \begin{equation*}
        \environment \in \concrete_2(\ab{\environment}) \implies \boolsem{b}(\environment) \in \concrete_{4a}(\absboolsem{b}(\ab{\environment}))
    \end{equation*}
\end{conjecture}


\begin{proof}
    \pfsketch\
    As above the proof would proceed by structural induction on $\mathbb{B}$
    In the base case, $\absboolsem{\texttt{true}}$ and $\absboolsem{\texttt{false}}$ would clearly be sound, further soundness of compare operations would follow an argument similar to the one above about operations.
    As a note, in the concrete semantics, compare operations between columns of tables is not allowed, we allow it in the abstract semantics in the case that a column would turn out to actually just be a singleton values in which case it would be allowed in the concrete semantics.
    In the inductive case, ignoring quantifiers for now, soundness would follow from the inductive hypothesis and a case analysis of each abstract Boolean operation.
    In the case of the quantifier statements, soundness follows from the inductive hypothesis, and the fact that we take the abstract and/or of all possible abstract values being quantified over, further we take care to add either $\true$ or $\false$ (for for all and exists respectively) to the result in the case that a value is \emph{missing} in the concrete counterpart.
\end{proof}


\begin{conjecture}
    \label{thm:sound-skip}
    $\abssem{\texttt{skip}}$ is sound, formally
    \begin{equation*}
    \environment \in \concrete_1(\ab{P}) \implies \sem{\texttt{skip}}(\environment) \subseteq \concrete_1(\abssem{\texttt{skip}}(\ab{P}))
    \end{equation*}
\end{conjecture}

\begin{proof}
    \pf\ Assume $\rho \in \concrete_1(\ab{P})$, then $\sem{\texttt{skip}}(\rho) = \{\rho\}$ and $\abssem{\texttt{skip}}(\ab{P}) = \ab{P}$, and therefore $\sem{\texttt{skip}}(\rho) \subseteq \concrete_1(\abssem{\texttt{skip}}(\ab{P}))$.
\end{proof}


\begin{conjecture}
    \label{thm:sound-assign}
    $\abssem{v_a \coloneqq e}$ is sound, formally
    \begin{equation*}
    \environment \in \concrete_1(\ab{P}) \implies \sem{v_a \coloneqq e}(\environment) \subseteq \concrete_1(\abssem{v_a \coloneqq e}(\ab{P}))
    \end{equation*}
\end{conjecture}


\begin{proof}
    \pf\
    Assume $\environment \in \concrete_1(\ab{P})$ then in $\sem{v_a \coloneqq e}(\environment)$, $v_a$ is assigned to $\expsem{e}(\environment)$.
    Because $\expsem{e}(\environment) \subset \concrete_{4a}(s) = \concrete_{4a}(\absexpsem{e}(\env{\ab{d}}, \env{\ab{a}}))$ for $(\env{\ab{d}}, \env{\ab{a}}) \in \ab{P}$ where $(\env{d}, \env{a}) \in \concrete_2(\env{\ab{d}}, \env{\ab{a}})$ then $\sem{v_a \coloneqq e}(\env{d}, \env{a}) \subseteq \concrete_2(\env{\ab{d}}, \env{\ab{a}}[v_a \mapsto s])$, therefore also $\sem{v_a \coloneqq e}(\env{d}, \env{a}) \subseteq \concrete_1(\abssem{v_a \coloneqq e}(\ab{P}))$.
\end{proof}


\begin{conjecture}
    \label{thm:sound-boolsem}
    $\abssem{b}$ is sound, formally
    \begin{equation*}
    (\rho_d, \rho_a)
        \in \concrete_1(\ab{P}) \implies \\
        \sem{b}(\rho_d, \rho_a) \subseteq \concrete_1(\abssem{b}(\ab{P}))
    \end{equation*}
\end{conjecture}


\begin{proof}
    \pf\
    Assume $(\env{d}, \env{a}) \in \concrete_1(\ab{P})$ and $\true \in \boolsem{b}(\env{d}, \env{a})$ then $\sem{b}(\env{d}, \env{a}) = (\env{d}, \env{a})$.
    Because $\boolsem{b}(\env{d}, \env{a}) \subseteq \concrete_{4a}(\absboolsem{b}(\env{\ab{d}}, \env{\ab{a}}))$ for $(\env{d}, \env{a}) \in \concrete_2(\env{\ab{d}}, \env{\ab{a}})$ where $(\env{\ab{d}}, \env{\ab{a}}) \in \ab{P}$ then $(\env{\ab{d}}, \env{\ab{a}})$ must be in $\abssem{b}(\ab{P})$ because $\absboolsem{b}(\env{\ab{d}}, \env{\ab{a}})$ must contain $B$ such that $B$ contains $\true$, because of the initial assumption.
\end{proof}


\begin{conjecture}
    \label{thm:sound-select}
    $\abssqlsem{\langle select(v_a, v_d, \mathbf{v}_t), \phi \rangle}$ is sound, formally
    \begin{multline*}
    (\rho_d, \rho_a)
        \in \concrete_2(\rho_{\ab{d}}, \rho_{\ab{a}}) \implies \\
        \sqlsem{\langle select(v_a, v_d, \mathbf{v}_t), \phi \rangle}(\rho_d, \rho_a) \subseteq \\
        \concrete_2(\abssqlsem{\langle select(v_a, v_d, \mathbf{v}_t), \phi \rangle}(\rho_{\ab{d}}, \rho_{\ab{a}}))
    \end{multline*}
\end{conjecture}


\begin{proof}
    \pfsketch\
    Soundness of $\abssqlsem{\langle select(v_a, \mathbf{v}_d), \phi \rangle}$ relies on the fact that we overestimate and select abstract tuples where we know the predicate $\phi$ is true, and also those where we are unsure.
    This selection is based on the abstract semantics of $\abspredsem{\phi}$, therefore if $\abspredsem{\phi}$ is sound so is $\abssqlsem{\langle select(v_a, \mathbf{v}_d), \phi \rangle}$.
\end{proof}


\begin{conjecture}
    \label{thm:sound-insert}
    $\abssqlsem{\langle insert(\mathbf{v}_d, \mathbf{e}), \phi \rangle}$ is sound, formally
    \begin{multline*}
    (\rho_d, \rho_a)
        \in \concrete_2(\rho_{\ab{d}}, \rho_{\ab{a}}) \implies \\
        \sqlsem{\langle insert(\mathbf{v}_d, \mathbf{e}), \phi \rangle}(\rho_d, \rho_a) \subseteq \\
        \concrete_2(\abssqlsem{\langle insert(\mathbf{v}_d, \mathbf{e}), \phi \rangle}(\rho_{\ab{d}}, \rho_{\ab{a}}))
    \end{multline*}

\end{conjecture}


\begin{proof}
    \pfsketch\
    Soundness of $\abssqlsem{\langle insert(\mathbf{v}_d, \mathbf{e}), \phi \rangle}$ relies on the fact that we take all possible combinations of the abstract input of the select statement.
    Therefore, as the evaluation of the input is based on $\absexpsem{\cdot}$ and because $\absexpsem{\cdot}$ is sound, $\abssqlsem{\langle insert(\mathbf{v}_d, \mathbf{e}), \phi \rangle}$ is sound.
\end{proof}


\begin{conjecture}
    \label{thm:sound-update}
    $\abssqlsem{\langle update(\absattrs, \absexps), \abspred \rangle}$ is sound, formally
    \begin{multline*}
    (\rho_d, \rho_a)
        \in \concrete_2(\rho_{\ab{d}}, \rho_{\ab{a}}) \implies \\
        \sqlsem{\langle update(\absattrs, \absexps), \abspred \rangle}(\rho_d, \rho_a) \subseteq \\
        \concrete_2(\abssqlsem{\langle update(\absattrs, \absexps), \abspred \rangle}(\rho_{\ab{d}}, \rho_{\ab{a}}))
    \end{multline*}
\end{conjecture}


\begin{proof}
    \pfsketch\
    Essentially $\abssqlsem{\langle update(\absattrs, \absexps), \abspred \rangle}$ is sound because:
    \begin{itemize}
        \item We do not update all the abstract tuples where we are sure the predicate evaluates to false;
        \item for all tuples where we are unsure whether the predicate is true we take the join of the incoming and current values being updated, effectively updating the values to both in an abstract sense;
        \item in the case where we are sure the predicate is true for the tuple is updated.
    \end{itemize}
    The soundness of the prequel relies on the fact that both that $\absexpsem{\cdot}$ is sound and that $\abspredsem{\cdot}$ is sound, which they are.
\end{proof}


\begin{conjecture}
    \label{thm:sound-delete}
    $\abssqlsem{\langle delete(\absattrs), \abspred \rangle}$ is sound, formally
    \begin{multline*}
    (\rho_d, \rho_a)
        \in \concrete_2(\rho_{\ab{d}}, \rho_{\ab{a}}) \implies \\
        \sqlsem{\langle delete(\absattrs), \abspred \rangle}(\rho_d, \rho_a) \subseteq \\
        \concrete_2(\abssqlsem{\langle delete(\absattrs), \abspred \rangle}(\rho_{\ab{d}}, \rho_{\ab{a}}))
    \end{multline*}
\end{conjecture}


\begin{proof}
    \pfsketch\
    The soundness of $\abssqlsem{\langle delete(\absattrs), \abspred \rangle}$ stems from the fact the only remove abstract tuples where we are absolutely sure that the predicate evaluates to true.
    The soundness of $\abssqlsem{\langle delete(\absattrs), \abspred \rangle}$ thus relies on the fact that $\abspredsem{\cdot}$ is a sound approximation.
\end{proof}


\begin{conjecture}
    \label{thm:sound-sql}
    $\abssem{C_{sql}}$ is sound, formally
    \begin{equation*}
    (\rho_d, \rho_a)
        \in \concrete_1(\ab{P}) \implies \sem{C_{sql}}(\rho_d, \rho_a) \subseteq \concrete_1(\abssem{C_{sql}}(\ab{P}))
    \end{equation*}
\end{conjecture}


\begin{proof}
    \pf\
    This follows from the fact that $\abssqlsem{C_{sql}}$ is sound for all $C_{sql}$.
\end{proof}


\begin{conjecture}
    $\abssem{I}$ is sound, formally
    \begin{equation*}
        (\rho_d, \rho_a)
            \in \concrete_1(\ab{P}) \implies \sem{I}(\rho_d, \rho_a) \subseteq \concrete_1(\abssem{I}(\ab{P}))
    \end{equation*}
\end{conjecture}


\begin{proof}
    \pf\
    This immediately follows from \autoref{thm:sound-skip},~\ref{thm:sound-assign},~\ref{thm:sound-boolsem} and~\ref{thm:sound-sql}.
\end{proof}
