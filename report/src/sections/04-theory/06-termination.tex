\subsection{Termination}\label{subsec:termination}

In this subsection we will argument for our analysis terminating.

Our analysis takes base in the constraint function seen in \autoref{eq:constraint}.
Hence, to prove that our analysis terminates, we would need to show that this function always reaches a fixed point.
Therefore, from \autoref{thm:kleene_finite} we need to show that the function is monotone and a complete and finite lattice.

As \autoref{eq:q} is defined over the abstract semantics, we need to show that $\abssem{I}$ is monotone, in order for it to be monotone.
Though, we have already showed in \autoref{thm:csql} that $\abssem{I}$ is monotone, hence \autoref{eq:q} is also monotone.

Now, looking at the type of \autoref{thm:csql}, we see that it maps to the tuples over the powerset of abstract environments.
So from \autoref{thm:kleene_finite}, we see that, in order for the analysis to reach a fixed point in a finite number of steps, $\bigtimes_{i = 1}^n \mathcal{P}(\ab{\mathfrak{E}})$ has to finite and complete.

To show that $\bigtimes_{i = 1}^n \mathcal{P}(\ab{\mathfrak{E}})$ is finite and complete, we must first show that $\ab{\mathfrak{E}}$ is finite.
To show that $\ab{\mathfrak{E}}$ is finite, we prove the following theorem:

\begin{restatable}{theorem}{absenv}\label{thm:absenv}
$\ab{\mathfrak{E}}$ is finite.
\end{restatable}

Now, from \autoref{thm:absenv} and \autoref{thm:powerset} we can conclude that the powerset $\mathcal{P}(\ab{\mathfrak{E}})$ is a finite and complete lattice.
Following this, from \autoref{thm:product-lattice} it also holds that $\bigtimes_{i = 1}^n \mathcal{P}(\ab{\mathfrak{E}})$ is a finite and complete lattice.
So, since our constraint function \autoref{eq:constraint} is a monotone function and $\bigtimes_{i = 1}^n \mathcal{P}(\ab{\mathfrak{E}})$ is a complete and finite lattice, we can conclude from \autoref{thm:kleene_finite} that our analysis reaches a fixed point in a finite number of steps, proving termination.