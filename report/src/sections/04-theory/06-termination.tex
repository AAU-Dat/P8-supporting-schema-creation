\subsection{Termination}\label{subsec:termination}

In this subsection we will argument for our analysis terminating.

Our analysis takes base in the semantics for instructions $\abssem{I}$.
Hence, to prove that our analysis terminates, we would need to show that $\abssem{I}$ always reaches a fixed point.

As seen in \autoref{eq:abssemtype}, $\abssem{I}$ maps to $\mathcal{P}(\ab{\mathfrak{E}})$.
We know from \autoref{thm:csql} that $\abssem{I}$ is a monotone function.
So from \autoref{thm:kleene_finite}, we see that, in order for the analysis to reach a fixed point in a finite number of steps, $\mathcal{P}(\ab{\mathfrak{E}})$ has to be defined over a finite and complete lattice.

To show that $\mathcal{P}(\ab{\mathfrak{E}})$ is a finite and complete lattice, we must show that $\ab{\mathfrak{E}}$ is a finite and complete lattice, as the powerset is merely all subsets.
To show that $\ab{\mathfrak{E}}$ is a finite and complete lattice, we prove the following theorem:

\begin{restatable}{theorem}{absenv}\label{thm:absenv}
$\ab{\mathfrak{E}}$ is a finite and complete lattice.
\end{restatable}

Now, from \autoref{thm:absenv} we can conclude that $\mathcal{P}(\ab{\mathfrak{E}})$ must also be a finite and complete lattice.
So, since $\abssem{I}$ is a monotone function and $\mathcal{P}(\ab{\mathfrak{E}})$ is a complete and finite lattice, we can conclude from \autoref{thm:kleene_finite} that our analysis reaches a fixed point in a finite number of steps, proving termination.